%---------
% Preamble
%---------
\documentclass[fleqn,12pt]{wlpeerj}
\usepackage{fixltx2e} % LaTeX patches, \textsubscript
\usepackage{cmap} % fix search and cut-and-paste in Acrobat
\usepackage{ifthen}
\usepackage[T1]{fontenc}
\usepackage[utf8]{inputenc}
\usepackage{hyperref}
\definecolor{orange}{cmyk}{0,0.4,0.8,0.2}
\definecolor{darkorange}{rgb}{.71,0.21,0.01}
\definecolor{darkblue}{rgb}{.01,0.21,0.71}
\definecolor{darkgreen}{rgb}{.1,.52,.09}
\hypersetup{pdftex,  % needed for pdflatex
  breaklinks=true,  % so long urls are correctly broken across lines
  colorlinks=true,
  urlcolor=blue,
  linkcolor=darkblue,
  citecolor=darkgreen,
  }

\setcounter{secnumdepth}{0}
\graphicspath{{../output/skimage/}}

% hyperlinks:
\ifthenelse{\isundefined{\hypersetup}}{
  \usepackage[colorlinks=true,linkcolor=blue,urlcolor=blue]{hyperref}
  \urlstyle{same} % normal text font (alternatives: tt, rm, sf)
}{}

\usepackage{caption}
\usepackage{subcaption}
\usepackage{float}

% Add line numbers globally for reviewers
\usepackage{lineno}
\linenumbers

\usepackage[gobble=auto]{pythontex}  % Must better syntax highlighting than listings

\PassOptionsToPackage{pdftex}{graphicx}
\usepackage{graphicx}  % Simplify graphics handling for figures


%-------------------------
% Begin scikit-image paper
%-------------------------

\begin{document}

\title{scikit-image: Image processing in Python}

% Author list affiliations
\author[1,2]{Stéfan van der Walt}
\affil[1]{Corresponding author: \protect\href{mailto:stefan@sun.ac.za}{stefan@sun.ac.za}}
\affil[2]{Stellenbosch University,
          Stellenbosch, South Africa}
\author[3]{Johannes L. Schönberger}
\affil[3]{Department of Computer Science,
          University of North Carolina at Chapel Hill,
          Chapel Hill, NC 27599, USA}
\author[4]{Juan Nunez-Iglesias}
\affil[4]{Victorian Life Sciences Computation Initiative,
          Carlton, VIC, 3010, Australia}
\author[5]{François Boulogne}
\affil[5]{Department of Mechanical and Aerospace Engineering,
          Princeton University,
          Princeton, New Jersey 08544, USA}
\author[6]{Joshua D. Warner}
\affil[6]{Department of Biomedical Engineering,
          Mayo Clinic,
          Rochester, Minnesota 55905, USA}
\author[7]{Neil Yager}
\affil[7]{AICBT Ltd,
          Oxford, UK}
\author[8]{Emmanuelle Gouillart}
\affil[8]{Joint Unit CNRS / Saint-Gobain,
          Cavaillon, France}
\author[9]{Tony Yu}
\affil[9]{Enthought Inc.,
          Austin, TX, USA}
\author[10]{and the scikit-image contributors}
\affil[10]{\url{https://github.com/scikit-image/scikit-image/graphs/contributors}}

% Key words to associate with the article
\keywords{image processing, reproducible research, education, visualization}

%---------
% Abstract
%---------

\begin{abstract}
  scikit-image is an image processing library that implements algorithms and utilities for use in research, education and industry applications. It is released under the liberal ``Modified BSD'' open source license, provides a well-documented API in the Python programming language, and is developed by an active, international team of collaborators. In this paper we highlight the advantages of open source to achieve the goals of the scikit-image library, and we showcase several real-world image processing applications that use scikit-image.
\end{abstract}

\flushbottom
\maketitle
\thispagestyle{empty}

%-----------------------------------------------------------------
% Sections of the paper are included in separate files with \input
%-----------------------------------------------------------------

% Introduction
\input{introduction}

% Getting started
\input{getting_started}

% Library contents
\input{library_contents}

% Data format and pipelining
\input{data_format_pipelining}

% Development practices
%----------------------
% Development practices
%----------------------

\section*{Development practices}
  \label{sec:development-practices}

  The purpose of scikit-image is to provide a high-quality library of powerful, diverse image processing tools free of charge and restrictions. These principles are the foundation for the development practices in the scikit-image community.

  The library is licensed under the \emph{Modified BSD license}, which allows unrestricted redistribution for any purpose as long as copyright notices and disclaimers of warranty are maintained \citep{BSD}. It is compatible with GPL licenses, so users of scikit-image can choose to make their code available under the GPL. However, unlike the GPL, it does not require users to open-source derivative work (BSD is not a so-called copyleft license). Thus, scikit-image can also be used in closed-source, commercial environments.

  The development team of scikit-image is an open community that collaborates on the \emph{GitHub} platform for issue tracking, code review, and release management\footnote{\url{https://github.com/scikit-image}}. \emph{Google Groups} is used as a public discussion forum for user support, community development, and announcements\footnote{\url{https://groups.google.com/group/scikit-image}}.

  scikit-image complies with the PEP8 coding style standard \citep{PEP8} and the NumPy documentation format \citep{NumpyDoc} in order to provide a consistent, familiar user experience across the library similar to other scientific Python packages. As mentioned earlier, the data representation used is \emph{n}-dimensional NumPy arrays, which ensures broad interoperability within the scientific Python ecosystem. The majority of the scikit-image API is intentionally designed as a functional interface which allows one to simply apply one function to the output of another. This modular approach also lowers the barrier of entry for new contributors, since one only needs to master a small part of the entire library in order to make an addition.

  We ensure high code quality by a thorough review process using the pull
  request interface on GitHub\footnote{\url{https://help.github.com/articles/using-pull-requests}, Accessed 2014-05-15.}.
  This enables the core developers and other interested parties to comment on
  specific lines of proposed code changes, and for the proponents of the
  changes to update their submission accordingly. Once all the changes have
  been approved, they can be merged automatically. This process applies not
  just to outside contributions, but also to the core developers.

  The source code is mainly written in Python, although certain performance critical sections are implemented in Cython, an optimising static compiler for Python \citep{Cython}. scikit-image aims to achieve full unit test coverage, which is above 87\% as of release 0.10 and continues to rise. A continuous integration system\footnote{\url{https://travis-ci.org}, \url{https://coveralls.io}, Accessed 2014-03-30} automatically checks each commit for unit test coverage and failures on both Python 2 and Python 3. Additionally, the code is analyzed by flake8 \citep{flake8} to ensure compliance with the PEP8 coding style standards \citep{PEP8}. Finally, the properties of each public function are documented thoroughly in an API reference guide, embedded as Python docstrings and accessible through the official project homepage or an interactive Python console. Short usage examples are typically included inside the docstrings, and new features are accompanied by longer, self-contained example scripts added to the narrative documentation and compiled to a gallery on the project website. We use Sphinx \citep{Sphinx} to automatically generate both library documentation and the website.

  The development master branch is fully functional at all times and can be obtained from GitHub. The community releases major updates as stable versions approximately every six months. Major releases include new features, while minor releases typically contain only bug fixes. Going forward, users will be notified about API-breaking changes through deprecation warnings for two full major releases before the changes are applied.


% Large section broken into separate files for each sub-section
\section*{Usage examples}
  \label{sec:usage-examples}

  % Usage in research
  \input{usage_research}

  % Usage in education
  \input{usage_education}

  % Usage in industry
  \input{usage_industry}

% Example: image registration and stitching
%------------------------------------------
% Example: image registration and stitching
%------------------------------------------

\section*{Example: image registration and stitching}
  \label{sec:example-image-registration-and-stitching}

  This section gives a step-by-step outline of how to perform panorama stitching using the primitives found in scikit-image. The full source code is at \url{https://github.com/scikit-image/scikit-image-demos}.

  \subsection{Data loading}
    \label{sub:data_loading}

    The ``ImageCollection'' class provides an easy way of representing multiple images on disk. For efficiency, images are not read until accessed.

    \begin{pyverbatim}
      from skimage import io
      ic = io.ImageCollection('data/*')
    \end{pyverbatim}

    Figure~\ref{fig:pano}(a) shows the Petra dataset, which displays the same facade from two different angles. For this demonstration, we will estimate a projective transformation that relates the two images. Since the outer parts of these photographs do not comform well to such a model, we select only the central parts. To further speed up the demonstration, images are downscaled to 25\% of their original size.

    \begin{pyverbatim}
      from skimage.color import rgb2gray
      from skimage import transform

      image0 = rgb2gray(ic[0][:, 500:500+1987, :])
      image1 = rgb2gray(ic[1][:, 500:500+1987, :])

      image0 = transform.rescale(image0, 0.25)
      image1 = transform.rescale(image1, 0.25)
    \end{pyverbatim}

  \subsection{Feature detection and matching}
    \label{sub:feature_detection_and_matching}

    ``Oriented FAST and rotated BRIEF'' (ORB) features \citep{ORB} are detected in both images. Each feature yields a binary descriptor; those are used to find the putative matches shown in Figure~\ref{fig:pano}(b).

    \begin{pyverbatim}
      from skimage.feature import ORB, match_descriptors

      orb = ORB(n_keypoints=1000, fast_threshold=0.05)

      orb.detect_and_extract(image0)
      keypoints1 = orb.keypoints
      descriptors1 = orb.descriptors

      orb.detect_and_extract(image1)
      keypoints2 = orb.keypoints
      descriptors2 = orb.descriptors

      matches12 = match_descriptors(descriptors1,
                                    descriptors2,
                                    cross_check=True)
    \end{pyverbatim}

  \subsection{Transform estimation}
    \label{sub:transform_estimation}

    To filter the matches, we apply RANdom SAmple Consensus (RANSAC) \citep{ransac}, a common method for outlier rejection. This iterative process estimates transformation models based on randomly chosen subsets of matches, finally selecting the model which corresponds best with the majority of matches. The new matches are shown in Figure~\ref{fig:pano}(c).

    \begin{pyverbatim}
      from skimage.measure import ransac

      # Select keypoints from the source (image to be
      # registered) and target (reference image).

      src = keypoints2[matches12[:, 1]][:, ::-1]
      dst = keypoints1[matches12[:, 0]][:, ::-1]

      model_robust, inliers = \
          ransac((src, dst), ProjectiveTransform,
                 min_samples=4, residual_threshold=2)
    \end{pyverbatim}

  \subsection{Warping}
    \label{sub:warping}

    Next, we produce the panorama itself. The first step is to find the shape of the output image by considering the extents of all warped images.

    \begin{pyverbatim}
      r, c = image1.shape[:2]

      # Note that transformations take coordinates in
      # (x, y) format, not (row, column), in order to be
      # consistent with most literature.
      corners = np.array([[0, 0],
                          [0, r],
                          [c, 0],
                          [c, r]])

      # Warp the image corners to their new positions.
      warped_corners = model_robust(corners)

      # Find the extents of both the reference image and
      # the warped target image.
      all_corners = np.vstack((warped_corners, corners))

      corner_min = np.min(all_corners, axis=0)
      corner_max = np.max(all_corners, axis=0)

      output_shape = (corner_max - corner_min)
      output_shape = np.ceil(output_shape[::-1])
    \end{pyverbatim}

    The images are now warped according to the estimated transformation model. Values outside the input images are set to -1 to distinguish the ``background''.

    A shift is added to ensure that both images are visible in their entirety. Note that \texttt{warp} takes the \textit{inverse} mapping as input.

    \begin{pyverbatim}
      from skimage.color import gray2rgb
      from skimage.exposure import rescale_intensity
      from skimage.transform import warp
      from skimage.transform import SimilarityTransform

      offset = SimilarityTransform(translation=-corner_min)

      image0_ = warp(image0, offset.inverse,
                     output_shape=output_shape, cval=-1)

      image1_ = warp(image1, (model_robust + offset).inverse,
                     output_shape=output_shape, cval=-1)
    \end{pyverbatim}

    An alpha channel is added to the warped images before merging them into a single image:

    \begin{pyverbatim}
      def add_alpha(image, background=-1):
          """Add an alpha layer to the image.

          The alpha layer is set to 1 for foreground
          and 0 for background.
          """
          rgb = gray2rgb(image)
          alpha = (image != background)
          return np.dstack((rgb, alpha))

      image0_alpha = add_alpha(image0_)
      image1_alpha = add_alpha(image1_)

      merged = (image0_alpha + image1_alpha)
      alpha = merged[..., 3]

      # The summed alpha layers give us an indication of
      # how many images were combined to make up each
      # pixel.  Divide by the number of images to get
      # an average.
      merged /= np.maximum(alpha, 1)[..., np.newaxis]
    \end{pyverbatim}

    The merged image is shown in Figure~\ref{fig:pano}(d). Note that, while the columns are well aligned, the color intensities at the boundaries are not well matched.

  % Combine multiple pano sub-figures elegantly with LaTeX
  \input{pano_multifig}

  \subsection{Blending}
    \label{sub:blending}

    To blend images smoothly we make use of the open source package Enblend \citep{Enblend}, which in turn employs multi-resolution splines and Laplacian pyramids \citep{burt_adelson_0,burt_adelson_1}. The final panorama is shown in Figure~\ref{fig:pano}(e).


% Discussion
\input{discussion}

% Conclusion
\input{conclusion}

% Acknowledgements
\input{acknowledgements}

% Finally, create the bibliograpy
\bibliography{skimage}

\end{document}
